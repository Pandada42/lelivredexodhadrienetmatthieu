\documentclass[12pt,a4paper]{report}
\usepackage[a4paper,margin=3cm]{geometry}

\usepackage[T1]{fontenc}
\usepackage[utf8]{inputenc}
\usepackage[french]{babel}
\usepackage{verse}
\usepackage{amsfonts}
\usepackage{amsthm}
\usepackage{amsmath}
\usepackage{amssymb}
\usepackage{hyperref}

\newtheorem{theorem}{Lemme}
\newtheorem{corollary}{Corollaire}  
\newtheorem{lemma}{Lemme}

\renewcommand\qedsymbol{$\blacksquare$}
\renewcommand{\d}{{\, \rm d}}

\title{Exercices}
\author{oui.}


\begin{document}
\maketitle
\tableofcontents
\section{Une Démonstration du Théorème de Cayley-Hamilton}\label{sec:cayley-analytique}
On se propose ici de démontrer le théorème de Cayley-Hamilton sur $\mathbb{C}$ par des méthodes analytiques.\\
On assimilera ici $\mathbb{C}$ et $\mathbb{C}I_{n}$ où $n \in \mathbb{N}$.\\
On prend $A \in M_{n}(\mathbb{C})$
\begin{enumerate}
    \item Montrer que pour $z \in \mathbb{C}$ suffisamment grand, $\det(z - A) \neq 0 $
    \item En déduire que pour $r$ assez grand, l'intégrale $\int\limits_{-\pi}^{+\pi}\frac{(re^{i\theta})^{k+1}}{re^{i\theta} - A}\frac{d\theta}{2\pi}$ a un sens.
\end{enumerate}

\section{Des Partitions d'un Ensemble fini}
On note $b_{n}$ le nombre de partitions d'un ensemble à $n$ éléments.
\begin{enumerate}
    \item Calculer $b_{0}, \dots, b_{3}$
    \item Trouver une relation de récurrence entre les $b_{n}$
    \item Exprimer $\sum\limits_{n=0}^{\infty}b_{n}\frac{x^{n}}{n!}$
    \item Donner une expression de $\b_{n}$
\end{enumerate}


\end{document}
